\documentclass{beamer}
%\documentclass[handout]{beamer}

\mode<presentation>
{% AnnArbor
%\usetheme{AnnArbor}
  %\usetheme{Boadilla}
  \usetheme{CambridgeUS}
 %\usetheme{Madrid}
  \setbeamercovered{transparent}
}

\usepackage{fontspec,xltxtra,xunicode,moreverb}

\usepackage[french]{babel}
% \usepackage{beamerthemesplit} // Activate for custom appearance

%% No navigation symbol.
\setbeamertemplate{navigation symbols}{}
\beamertemplatenavigationsymbolsempty

%\setbeameroption{hide notes}

%\newcommand{\mypause}{\pause}
\newcommand{\mypause}{~}

\newcommand{\elvrm}{\rm}
\newcommand{\fivrm}{\rm}
\newcommand{\sixrm}{\rm}
\newcommand{\sevrm}{\rm}
\newcommand{\egtrm}{\rm}
\newcommand{\ninrm}{\rm}
\newcommand{\tenrm}{\rm}
\newcommand{\twlrm}{\rm}
\newcommand{\frtnrm}{\rm}
\newcommand{\svtnrm}{\rm}
\newcommand{\twtyrm}{\rm}
\newcommand{\twfvrm}{\rm}

\newcommand{\ex}{{\bf Exemple}}
\newcommand{\afor}{\bf for}
\newcommand{\ato}{\bf to}
\newcommand{\ado}{\bf do}
\newcommand{\aendo}{\bf endo}
\newcommand{\esp}{\hspace{0.5cm}}

\newcommand{\sal}{\sum_i \mu_i}
\newcommand{\sbet}{\sum_j \nu_j}
\newcommand{\If}{\mbox{\bf if }}
\newcommand{\Then}{\,\mbox{\bf then }}
\newcommand{\titre}[1]{\title{{{\color{red} \large \bf #1}}}}

% Needed for course 2. 
\newcommand{\Zn}{{\bf Z}^n}
\newcommand{\Z}{{\bf Z}}
\newcommand{\N}{{\bf N}}
\newcommand{\Np}{{\bf N}^p}

\newcommand{\alfa}{\textsc{alpha}}
\newcommand{\alfacase}{\mbox{\bf case}}
\newcommand{\alfaesac}{\mbox{\bf esac}}
\newcommand{\zset}{\mathbb{Z}}

\newcommand{\sys}{{\bf system}}
\newcommand{\real}{{\bf real}}
\newcommand{\of}{{\bf of}}
\newcommand{\lett}{{\bf let}}
\newcommand{\tel}{{\bf tel}}
\newcommand{\returns}{{\bf returns}}
\newcommand{\boolean}{{\bf boolean}}
\newcommand{\true}{{\bf true}}
\newcommand{\false}{{\bf false}}

\newcommand{\pomme}{\texttt{cmd}}
\newcommand{\alalign}{{$\hookleftarrow$}}

\newcommand{\pyth}{{\sc Python}}
\newcommand{\prog}[1]{\alert{\texttt{#1}}}

\title{Introduction à l'informatique, avec \pyth{}}
% \author{Patrice Quinton}
\author{Lilian Besson}
\date{2020--2021\\Version du \today}

\institute% (optional, but mostly needed)
{ENS Rennes}

\date%[\today] % (optional, should be abbreviation of conference name)
[Info -- DEM -- 2020]{Initiation à l'informatique -- DEM -- 2020\\Module 4}
\logo{\includegraphics[height=0.5cm]{logoENS.pdf}}

% Delete this, if you do not want the table of contents to pop up at
% the beginning of each subsection:
\AtBeginSection[]
{
  \begin{frame}<beamer>
    \frametitle{Outline}
%    \tableofcontents[currentsection,currentsubsection]
\tableofcontents[currentsection,currentsubsection,hideallsubsections]
  \end{frame}
}

\begin{document}

\frame{\titlepage}

\section[Outline]{}
\frame{\tableofcontents}
\frame
{
\frametitle{Résumé des épisodes précédents}
{
On a vu:
  \begin{itemize}
  \item comment accéder à l'ordinateur via son \prog{terminal}
  \item quelques commandes simples permettant de se déplacer dans la \prog{hiérarchie des fichiers}
  \item comment démarrer \pyth{}
  \item comment écrire un programme avec l'éditeur \prog{idle}
  \item les \prog{variables}, leur \prog{type}, la boucle \prog{while}, l'affectation etc.
  \item les \prog{conditions} et instructions conditionnelles
  \item les \prog{listes}
  \end{itemize}
La suite:
  \begin{itemize}
  \item Définir des \prog{fonctions}
  \item Préparer un petit projet
  \end{itemize}
}
}

\section{Les fonctions}
\frame
{
\frametitle{Fonctions}
{\footnotesize
\begin{block}{Définir une fonction}\mypause{}
Jusqu'ici, \pyth{} n'était pour nous qu'une calculette
pas nécessairement facile à utiliser. \mypause{}
\begin{itemize}
\item Dans votre fenêtre éditeur (\prog{idle})\mypause{}
\item Menu \prog{File}, commande \prog{New file}.
Le résultat est une nouvelle fenêtre, appelée \prog{untitled}.\mypause{}
\item Dans cette fenêtre, on va taper des lignes de \pyth{}, permettant de fabriquer une
\alert{\em fonction}\mypause{}
\item Tapez \prog{def mafonction(x,y):}, puis \alalign{}\mypause{}
\item Tapez \prog{return(x + 2*y + 5)}, puis \alalign{}\mypause{}
\item \alert{Sauvegardez} ce programme en lui donnant le nom de fichier \prog{mafonction}\mypause{}
\item Vérifier cette fonction avec la commande \texttt{Check} puis, lorsque c'est bon,
utilisez la commande \texttt{Run module}
\end{itemize}
\end{block}
}
}

\frame
{
\frametitle{Fonctions}
{\footnotesize
\begin{block}{Utiliser une fonction}
Que se passe-t-il ?\mypause{}
\begin{itemize}
\item Au début, pas grand-chose. Vous remarquerez que \pyth{} a été relancé.\mypause{}
\item Mais tapez \prog{mafonction(2,3)} et \alalign{}\mypause{}
\item Vous devez obtenir un résultat!\mypause{}
\item L'\alert{environnement} de \pyth{} s'est enrichi d'une nouvelle fonction\mypause{}
\item Tapez \prog{type(mafonction)}: \pyth{} vous dit que \texttt{mafonction}
est un objet de type \prog{'function'}, c'est-à-dire, une fonction\mypause{}
\item Utilisez votre fonction avec diverses valeurs\mypause{}
\item Notez que si vous redémarrez \pyth{} (comment faire?), il ne connaîtra pas 
votre fonction, sauf si vous utilisez \prog{Run module}\mypause{}
\end{itemize}
\end{block}
}
}

\frame{
\frametitle{Le format d'une fonction}
{\footnotesize
\begin{block}{Eléments de base}\mypause{}
\begin{itemize}
\item \prog{def nom ( x, y, z... ):} C'est la \alert{{\em signature de la fonction}}, 
avec son nom et ses paramètres (ou ses arguments). \mypause{}
\item La définition de la fonction apparaît ensuite, sur plusieurs lignes {\em décalées}
par une tabulation (on dit aussi {\em indentées}). \mypause{}
\item Ces lignes indentées forment \alert{{\em le corps}} de la fonction, qui 
forme aussi ce qu'on appelle \alert{{\em un bloc}}\mypause{}
\item On peut y définir autant de variables que l'on veut: ces variables
sont \alert{privées} au bloc: en dehors du bloc, elles n'existent plus\mypause{}
\item On peut associer à une fonction une \alert{{\em aide en ligne}}\mypause{}
\item On peut utiliser des commentaires, qui commencent par le caractère \prog{\#}
\end{itemize}
\end{block}
}
}

\frame{
\frametitle{Exemple}
{\footnotesize
%\begin{block}{Exemple}
\alert{\verbatiminput{../Programmes-python/ex1.py}}
%\end{block}
}
}

\frame
{
\frametitle{Quelques remarques}
{\footnotesize
\begin{block}{D'une fenêtre à l'autre}\mypause{}
\begin{itemize}
\item
Vous travaillez sans doute avec trois fenêtres simultanément: l'éditeur \prog{idle},
une fenêtre \pyth{}, et votre terminal.\mypause{}
\item
Avec l'éditeur, ce que vous tapez peut être sauvegardé dans un fichier 
dont vous avez choisi le nom, par exemple, \prog{monProgramme.py}.
Ayez le réflexe de sauvegarder votre programme très très souvent (à chaque fois
que vous avez tapé quelques lignes), en utilisant le raccourci (en général, \prog{cmd S}).\mypause{}
\item
Lorsque vous voulez exécuter votre programme, sauvegardez-le, puis activez
la commande \prog{Run module}: elle vous place dans la fenêtre \pyth{}, relance
l'interpréteur (qui oublie alors ce qui avait été fait précédemment).\mypause{}
\item
Dans le fenêtre \prog{pyth}, vous pouvez aussi taper directement des 
fragments de programme, par exemple, pour bien comprendre ce que 
fait une instruction. 
\end{itemize}
\end{block}
}
}

\frame
{
\frametitle{Quelques remarques}
{\footnotesize
\begin{block}{Aide en ligne, apprentissage}\mypause{}
\begin{itemize}
\item Les notions essentielles à la programmation -- variables, conditions, fonctions, entiers, algorithmes, etc. --
sont \alert{\em indépendantes} du langage de programmation que l'on utilise.\mypause{}
\item Elles sont \alert{\em peu nombreuses}.\mypause{}
\item Les fonctions que \pyth{} met à votre disposition sont nombreuses. Vous pouvez y accéder
grâce à l'aide en ligne (menu \prog{help}). C'est intéressant, mais assez chronophage.\mypause{}
\item Les librairies disponibles pour les principaux langages de programmation sont \alert{\em immenses}.
C'est le cas pour \pyth{} et cela fait son intérêt: des modules entiers sont développés et deviennent
accessibles gratuitement (logiciel libre).\mypause{}
\item Lorsqu'on veut réaliser une application significative, on commence par rechercher les 
librairies dont on pourrait déjà bénéficier. \mypause{}
\item À la fin de cette petite initiation, vous serez en mesure de développer, avec de la pratique, 
des programmes en \pyth{}. 
\end{itemize}
\end{block}
}
}

\frame{
\frametitle{Exercice M4-1}
{\footnotesize
\begin{block}{Faire}
  Reprendre le programme qui calcule la somme des éléments d'une liste, et
  le transformer en une fonction.
\end{block}

\begin{block}{Rappel}
\prog{
\verbatiminput{../../Cours-3/Programmes-Python/M3-4.py}
}

\end{block}
}
}

\frame{
\frametitle{Exercice M4-1}
{\footnotesize
\begin{block}{Faire}
  Reprendre le programme qui calcule la somme des éléments d'une liste, et
  le transformer en une fonction.
\end{block}

\begin{block}{Solution}
\prog{
\verbatiminput{../Programmes-Python/M4-1.py}
}

\end{block}
}
}


\frame{
\frametitle{En résumé}
\begin{itemize}
\item On a vu \alert{comment préparer un programme \pyth{}}
\item Comment en \alert{vérifier la syntaxe} (\texttt{check})
\item Comment le \alert{sauvegarder}
\item Comment lancer son \alert{exécution}
\item Comment utiliser des \alert{entiers}, des \alert{listes d'entiers}
\item Comment définir des \alert{fonctions}
\end{itemize}

}

\end{document}
