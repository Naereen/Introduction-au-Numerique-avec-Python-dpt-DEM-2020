% XeLaTeX can use any Mac OS X font. See the setromanfont command below.
% Input to XeLaTeX is full Unicode, so Unicode characters can be typed directly into the source.

% The next lines tell TeXShop to typeset with xelatex, and to open and save the source with Unicode encoding.

%!TEX TS-program = xelatex
%!TEX encoding = UTF-8 Unicode

\documentclass[12pt]{article}

\usepackage{geometry}                % See geometry.pdf to learn the layout options. There are lots.
\geometry{letterpaper}                   % ... or a4paper or a5paper or ...
%\geometry{landscape}                % Activate for for rotated page geometry
%\usepackage[parfill]{parskip}    % Activate to begin paragraphs with an empty line rather than an indent
\usepackage{graphicx}
\usepackage{amssymb}

% Will Robertson's fontspec.sty can be used to simplify font choices.
% To experiment, open /Applications/Font Book to examine the fonts provided on Mac OS X,
% and change "Hoefler Text" to any of these choices.

\usepackage{fontspec,xltxtra,xunicode}
%\usepackage{array}
% \defaultfontfeatures{Mapping=tex-text}
% \setromanfont[Mapping=tex-text]{Hoefler Text}
% \setsansfont[Scale=MatchLowercase,Mapping=tex-text]{Gill Sans}
% \setmonofont[Scale=MatchLowercase]{Andale Mono}

\newcommand{\prog}[1]{\texttt{#1}}

\title{DEM-1 -- Introduction au Numérique\\
Questionnaire d'auto-évaluation}
% \author{Auto-évaluation}
\date{16 octobre 2020}  % Activate to display a given date or no date

\begin{document}
\maketitle


\section{Votre nom et votre prénom}
\begin{center}
\begin{tabular}{| p{4cm} | p{4cm} | p{5cm} |}
\hline
Prénom & Nom & mél\\
\hline
~&&\\
\hline
\end{tabular}
\end{center}
Merci de m'indiquer le nombre de cours que vous avez
"ratés" :
\begin{center}
\begin{tabular}{| p{7cm} | p{1cm} |}
\hline
Nombre de cours ratés : & \\
\hline
\end{tabular}
\end{center}

\section{Méthode}

\newcommand{\pyth}{{\sc Python}}

L'auto-évaluation est faite de la façon suivante :
\begin{itemize}
\item Donnez-vous une première note entre 0 et 20, selon la grille d'évaluation
qui vous est donnée dans la section~\ref{note1}
\item Expliquez les raisons (en quelques lignes) de cette évaluation.
\item Répondez avec honnêteté au questionnaire
de la section~\ref{note2}. Vous mettez 1 à une question si vous
estimez avoir la compétence correspondante, 0 sinon, 1/2 si vous n'êtes pas sûr-e.\\
(NB : il ne vous est \textbf{pas} demandé de justifier vos réponses !)
\item Totaliser et rapporter la note obtenue
\item Répondre au questionnaire d'évaluation du cours de la section~\ref{eval}.
\end{itemize}

\newpage

\section{Votre première note}
\label{note1}

\begin{center}
\begin{tabular}{| c | p{12cm} |}
\hline
Note & Signification\\
\hline
18 & Je suis parfaitement à l'aise, et j'aurais pu effectuer tous les exemples de programmation sans difficulté\\
\hline
16 & J'aurais pu effectuer tous les exemples proposés, à condition d'y passer quelque
temps\\
\hline
14 & J'ai bien compris les notions essentielles, je peux refaire des exemples de programmation
similaires\\
\hline
12 & J'ai assez bien compris les notions présentées, je peux refaire les exem\-ples les plus
simples\\
\hline
10 & Je n'ai compris qu'une partie de ce qui a été présenté en cours\\
\hline
8 & Je n'ai pas compris les exemples présentés, et je ne serais pas capable
de les refaire\\
\hline
6 & Je ne sais pas comment écrire un tout petit programme avec \pyth{} ni
comment l'exécuter\\
\hline
4 & Je ne sais pas comment ouvrir un terminal sur mon ordinateur personnel\\
\hline
\end{tabular}
\end{center}

\begin{center}
\begin{tabular}{| p{7cm} | p{1cm} |}
\hline
La note que vous vous donnez : & \\
\hline
\end{tabular}
\end{center}

\begin{center}
\begin{tabular}{| p{14cm} |}
\hline
Commentaire\\
\hline
~\\
\vspace{3cm}
\\
\hline
\end{tabular}
\end{center}

\section{Le questionnaire}
\label{note2}
\begin{enumerate}
\item $\Box$ Savez-vous vous connecter à votre ordinateur et lancer le terminal ?
\item $\Box$ Savez-vous ce qu'est un répertoire ? Savez-vous trouver dans quel répertoire vous êtes lorsque vous avez
lancé le terminal ?
\item $\Box$ Savez-vous lister les fichiers du répertoire courant à partir du terminal ?
\item $\Box$ Pouvez-vous indiquer le type des principaux fichiers qui sont dans votre répertoire ?
\item $\Box$ Savez-vous comment créer un nouveau répertoire à partir du terminal ?
\item $\Box$ Savez-vous comment vous déplacer d'un répertoire à l'autre à partir du terminal ?
\item $\Box$ Savez-vous déplacer ou détruire un fichier à partir du terminal ?
\item $\Box$ Savez-vous lancer le programme \pyth{} (\prog{python}) ?
\item $\Box$ Connaissez-vous la différence entre un interpréteur et un compilateur ?
\item $\Box$ Pouvez-vous écrire un petit programme \pyth{} à qui on donne deux nombres (avec \prog{input}) et qui en affiche la somme ?
\item $\Box$ Avez-vous compris ce qu'est une \emph{condition} ?
\item $\Box$ Avez-vous compris ce qu'est une \emph{expression booléenne}, et savez vous
écrire de telles expressions (lorsqu'elles sont simples) ?
\item $\Box$ Avez-vous compris ce qu'est un \emph{bloc} dans un programme \pyth{} ?
\item $\Box$ Avez-vous compris ce qu'est une \emph{fonction} ?
\item $\Box$ Savez-vous ce qu'est une \emph{liste}, comment accéder à ses éléments, à des sous-listes, à en modifier un élément ?
\item $\Box$ Sauriez-vous expliquer l'algorithme permettant de ranger des crêpes (sans le programmer) ?
\item $\Box$ Sauriez-vous expliquer comment on peut choisir de représenter une pile de crêpe en \pyth{}
\item $\Box$ Pouvez-vous écrire un petit programme qui lit un texte (avec \prog{input}), et compte son nombre de voyelles ?
\item $\Box$ Avez-vous compris ce qu'est une librairie et à quoi cela sert ?
\item $\Box$ Avez-vous compris les relations -- similitudes, différences -- qui existent entre un langage informatique et une
langue naturelle ?
\end{enumerate}

\begin{center}
\begin{tabular}{| p{3cm} | p{1cm} |}
\hline
Total& \\
\hline
\end{tabular}
\end{center}

\section{\'Evaluation du cours}
\label{eval}

Ce petit questionnaire est destiné à améliorer le cours.\\
(Entourez la réponse que vous choisissez.)

\begin{center}
\begin{tabular}{| p{5cm} | p{3cm} | p{3cm} | p{3cm} |}
\hline
\textbf{Question} & & & \\
\hline
La progression du cours est & convenable & trop rapide & trop lente\\
\hline
Le contenu du cours est & utile & inutile & pas d'avis \\
\hline
L'initiation à un langage de programmation est & utile & inutile & pas d'avis \\
\hline
Retransmettre le cours sur BigBlueButton était & utile & inutile & pas d'avis \\
\hline
Les "petites histoires" vous semblent & intéressantes & inintéressantes & pas d'avis \\
\hline
\end{tabular}
\end{center}

\begin{center}
\begin{tabular}{| p{15cm} |}
\hline
Commentaire (améliorations possibles pour l'an prochain !)\\
\hline
~\\
\vspace{8cm}
\\
\hline
\end{tabular}
\end{center}


\end{document}